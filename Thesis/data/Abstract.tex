\TAMUAbstractFormat
%%%%%%%%%%%%%%%%%%%%%%%%%%%%%%%%%
%Please keep the line above and only type in below.
%%%%%%%%%%%%%%%%%%%%%%%%%%%%%%%%%

In various situations, there is a need to estimate the number of active devices within a specific area.
This thesis offers one possible approach to accomplish this task.
It focuses on estimating the number of devices in a certain area based on monitoring and processing Wi-Fi metadata which includes the Received Signal Strength Indicator.
To accomplish this goal, four sensing devices are placed at the corners of a rectangular area.
These sensing devices observe and record local data traffic, along with the received signal strength associated with each packets.
For each sensing device, two types of frontends are considered, namely directional and isotropic antennas.
Each sensing device retrieves the received signal strength indicators and the media access control addresses from the 802.11 frames packets transmitted by nearby active wireless devices.
The estimator takes the received signal strength indicators as input and infers the number of active Wi-Fi devices inside the area of interest.
Two algorithms, bayesian and maximum-likelihood, are employed for estimation purposes.
Overall performance is used to compare and contrast the systems implemented with directional antennas and isotropic antennas, respectively.
Theoretical and experimental results both hint at performance improvements when using directional antennas, when compare to standard isotropic antennas.

