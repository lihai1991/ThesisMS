\chapter[INTRODUCTION]{INTRODUCTION}

The number of Wi-Fi access points and the number of Wi-Fi client devices have been increasing dramatically in recent years.
Cisco Systems predicts in their Visual Network Index that 55~percent of total mobile data traffic will be offloaded onto fixed networks through Wi-Fi access points and femtocells by 2020~\cite{CiscoVNI2016}.
Modern smartphones equipped with Wi-Fi modules transmit Wi-Fi messages periodically.
Therefore, this provides a means to estimate occupancy by passively listening to Wi-Fi packets.
More specifically, by deploying Wi-Fi monitoring devices in an area of interest, it is possible to detect these Wi-Fi transmissions.
Each acquired Wi-Fi packet contains a unique MAC address.
This information can be augmented by the received signal strength indicator (RSSI) of the captured signal.
In the current context, the MAC address serves as a device identifier, whereas the RSSI provides partial information about the physical distance between the transmitter and the receiver.

In the research, we focus on occupancy estimation based on Wi-Fi packets and we analyze the benefits associated with using directional antennas.
We introduce a Bayes estimation algorithm and a maximum likelihood scheme to estimate occupancy.
We employ numerical simulations to compare the performance corresponding to sensing devices with directional antennas and isotropic antennas.
In addition, our findings are further supported through field experimentation.
The testbed is implemented in a line-of-sight environment, with four sensing devices deployed at the corners of the area of interest.

The remainder of this proposal is organized as follows.
In Section~2, we describe related work.
In Section~3, we introduce a formal problem formulation.
In Section~4, we propose two estimation schemes.
These schemes are evaluated through numerical simulations in Section~5.
We then discuss experimental result, along with description of experiment setup.
Finally, we offer concluding remarks in the last section.

 






 


