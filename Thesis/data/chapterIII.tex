\chapter{SYSTEM MODEL AND PROBLEM FORMULATION}
Wi-Fi based occupancy estimation is attractive because it only require simple sensing device Meanwhile it can benefit from local Wi-Fi network and the high penetration rate of smartphone. The existence of tools like pcap and wireshark makes monitoring Wi-Fi environment a straight forward work. In our study, we will extract the RSSI and MAC address for the purpose of occupancy estimation. As is discussed above, the RSSI is related to the distance between Rx and Tx. Therefore we can infer the location information by RSSI. However RSSI does not only depend on distance, some other factors like noise and fading will also influence it. As such, a proper wireless channel model is needed to take care of this situation.
\section{Wireless Channel Model}
A common wireless environment can be express by
\begin{equation*}
r (t) = g (d) s (t) + w (t)
\end{equation*}
where $r (t)$ represents received signal. $g (d)$ the function of distance is the power gain which is related to several factors including the mean path loss, shadow fading and antenna gain. $s (t)$ denotes the sent signal and $w (t)$ is additive white Gaussian noise.
Here we use the log-normal channel model. So the received power for a given distance between transmitter and receiver can be expressed as
\begin{equation}
P_{d} \text{[dBm]}
= A + B \log_{10}(d) + L_{s} + G_{a} 
\end{equation}
where $A$ is a combination of the transmitted signal power and average path loss
and $B$ represents the path loss coefficient.
$L_{s}$ is an independent and identically distributed Gaussian random variable which represents shadowing.
$G_{a}$ is the antenna gain. Note that the mean of $L_{s}$ can be added in $A$, thus we assume $L_{s}$ is a zero mean random variable, $L_{s}\sim\mathcal{N}(0,\sigma_{\mathrm{s}}^2)$. $\sigma_{s}$ is the variance of shadowing component. Therefore in the logarithmic domain, the density function of $L_{s}$ can be written as
\begin{equation*}
f_{L_{s}} (\ell)
= \frac{1}{\sqrt{2 \pi} \sigma_{\mathrm{s}}} 
\exp \left( - \frac{\ell^2}{2 \sigma_{\mathrm{s}}^2} \right) 
\end{equation*}
Here the variance can be estimated by sample data set. An unbiased estimator for the variance is given by~\cite{zwillinger1995crc}
\begin{equation*}
\sigma_{\mathrm{s}}^2 = \frac{1}{N-1} \sum_{1}^{N} (l_{k}-\mu_{s})^2
\end{equation*} 

Consider a scenario where several wireless devices are randomly positioned nearby a rectangular area of interest.
Four monitoring devices are located at the corners of this region.
Each monitoring device has information concerning its own location and orientation.
The radiation pattern of antenna attached to each monitoring device is known as well.
All of the monitoring devices are connected to the Internet and send the captured data to a process center for inference.
The wireless clients transmit data packets periodically and consequently they can be easily detected by the monitoring devices.
Since each wireless client has a unique MAC address, the packets transmitted from different clients can be distinguished.
Throughout, we use $\mathcal{A}_{\mathrm{t}}$ to represent the target area and $\mathcal{A}_{\mathrm{o}}$ to represent its complement.
In this study, we assume the wireless clients are quasi-static and each client is equipped with an isotropic antenna.
For convenience, we use a single vector to denote the locations of the wireless clients.
\begin{equation}
\underline{\mathbf{U}} = (\mathbf{U}_1, \ldots, \mathbf{U}_{n_{\mathrm{a}}}).
\end{equation}
where $n_{\mathrm{a}}$ is the number of the detected clients.
We also assume that the signal captured by a monitoring device comes from a line-of-sight path.
Therefore, signal strength subscribes to a free-space transmission model.
The received signal strength from client $j$ to sensing device $i$ can be expressed as
\begin{equation} 
P_{ij} \text{[dBm]}
= A + B \log_{10}(d_{ij}) + L_{ij} + G_i (\phi_{ij})
\end{equation}
where $A$ and $B$ are the mean decay parameters, $d_{ij}$ is the Euclidean distance between the client~$j$ and sensing device~$i$.
$L_{ij}$ represent shadow fading and $G_i (\cdot)$ is the antenna gain function of the sensing device.
Parameter $\phi_{ij}$ denotes the angle of the signal transmission direction.
The shadow fading components $\{ L_{ij} \}$ are assumed to be independent and identically log-normal distributed random variables.
In the logarithmic domain, the probability density function is
\begin{equation} 
f_{L_{ij}} (\ell)
= \frac{1}{\sqrt{2 \pi} \sigma_{\mathrm{s}}} 
\exp \left( - \frac{\ell^2}{2 \sigma_{\mathrm{s}}^2} \right)
\end{equation}
where $\sigma_{\mathrm{s}}$ is the standard deviation of shadowing.
The observed information from the four sensing devices form a power matrix $\underline{\mathbf{P}} = (\mathbf{P}_1, \ldots, \mathbf{P}_{n_{\mathrm{a}}})$.
The vector element $\mathbf{P}_j = (P_{1j}, P_{2j},P_{3j},P_{4j})$ contains signal strength of wireless client~$j$ detected by four sensing devices.
We assume the number and locations of wireless clients located inside the area of interest form a Poisson point process with intensity $\lambda_{\mathrm{t}}$.
Therefore,
\begin{equation*}
\Pr ( R_{\mathrm{t}} = r_{\mathrm{t}} )
= \frac{(\lambda_{\mathrm{t}} A_{\mathrm{t}})^{r_{\mathrm{t}}}}
{r_{\mathrm{t}}!} e^{- A_{\mathrm{t}} \lambda_{\mathrm{t}}}
\quad r_{\mathrm{t}} = 0, 1, \ldots
\end{equation*}
where $R_{\mathrm{t}}$ is the number of clients inside.
$A_{\mathrm{t}}$ is the area of the target region.
Similarly, we get
\begin{equation*}
\Pr ( R_{\mathrm{o}} = r_{\mathrm{o}} )
= \frac{(\lambda_{\mathrm{o}} A_{\mathrm{o}})^{r_{\mathrm{o}}}}
{r_{\mathrm{o}}!} e^{- A_{\mathrm{o}} \lambda_{\mathrm{o}}}
\quad r_{\mathrm{o}} = 0, 1, \ldots
\end{equation*}
where $R_{\mathrm{o}}$ is the number of clients outside.
$A_{\mathrm{o}}$ is the area of the complimentary of target region.
$\lambda_{\mathrm{o}}$ is a Poisson intensity parameter.
The inference task is to estimate occupancy based on the power matrix $\underline{\mathbf{P}}$.
\ignore{\section{Section Test Example 1 }


Test section for TOC display

\section{Test Section in this Chapter}

Section Title is to test toc display only, no actual meaning.

\subsection{Test Subsection in this Chapter}
Test subsection for TOC display

\subsection{Subsection Test Example 1}

Test subsection for TOC display

\section{Section Test Example 2}

Test section for TOC display

\subsection{Test Subsection in this Chapter}

Test subsection for TOC display

\subsection{Subsection Test Example 3 }
Test subsection for TOC display

Example of multiple line \textbf{verbatim} environment.

\begin{lstlisting}
  123
x 456
=============
  738 (this is 123 x 6)
 615  (this is 123 x 5, shifted one position to the left)
+492  (this is 123 x 4, shifted one position to the left)
=============
56088
\end{lstlisting}



\subsection*{Subsubsection Test Example 2}

some text here(the text will not be displayed in the TOC)


\subsection{Subsection Test Example 4}
Test subsection for TOC display

\subsection{Section Summary}
  
\begin{lstlisting}[language={Verilog},tabsize=5,title={A simple paragraph of Verilog code is below in verbatim}]
  Always@(posedge LRCK)
	Begin
	Counter = Counter + 1
	If (Counter == 40)
	Begin
	Counter = 0
	Phase_control_word = Phase_Control_word + 1
	If (phase_control_word >= 7168)
	Begin
	Phase_control_word =4778
	end
	end
	end
\end{lstlisting}

\section{Section Test Example 3}
Test section for toc display only

\subsection{Subsection Test 1}
Test subsection for toc display only.

\subsection{Subsection Test 2}
Test subsection for toc display only.

\subsection{Subsection Test 3}
Test subsection for toc display only.

\subsection{Subsection Test 4}
Test subsection for toc display only.

\section{Section Test Example 4}
Test section for toc display only}






