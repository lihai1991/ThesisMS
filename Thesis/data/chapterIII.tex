\chapter{SYSTEM MODEL AND PROBLEM FORMULATION}

Wi-Fi based occupancy estimation is attractive because it only require simple sensing devices.
Meanwhile, it can benefit from local Wi-Fi network and the high penetration rate of smartphones.
The existence of tools like \texttt{pcap} and wireshark makes monitoring Wi-Fi environments a straightforward task.
In this thesis, we extract the RSSI and MAC address for the purpose of occupancy estimation.
As discussed above, the RSSI is related to the distance between the Rx and Tx.
Thus, we can estimate the location information based on RSSI values.
Still, RSSI does not depend only on distance; some other factors like noise and fading can also influence it.
As such, a proper wireless channel model is needed to take care of this situation.


\section{Wireless Channel Model}
\label{section:channel}

A common wireless environment can be express as
\begin{equation*}
r (t) = g (d) s (t) + w (t) ,
\end{equation*}
where $r (t)$ represents the received signal; the function $g (d)$ is the power gain which is related to several factors including the mean path loss, shadow fading and antenna gain; $s (t)$ denotes the sent signal; and $w (t)$ is noise.
Herein, we adopt the log-normal channel model.
So, the expected received power for a given distance between transmitter and receiver can be expressed as
\begin{equation} \label{channel}
P_{d} \text{[dBm]}
= A + B \log_{10}(d) + L_{s} + G_{a} ,
\end{equation}
where $A$ is a combination of the transmitted signal power and average path loss
and $B$ represents the path loss coefficient.
Component $L_{s}$ is a Gaussian random variable that captures shadowing.
Finally, $G_{a}$ is the antenna gain.
We note that the mean of $L_{s}$ can be added in $A$; thus, we assume $L_{s}$ is a zero mean random variable, $L_{s} \sim \mathcal{N}(0,\sigma_{\mathrm{s}}^2)$.
Parameter $\sigma_{s}$ is the variance of the shadowing component.
In the logarithmic domain, the probability density function of $L_{s}$ can be written as
\begin{equation*}
f_{L_{s}} (\ell)
= \frac{1}{\sqrt{2 \pi} \sigma_{\mathrm{s}}} 
\exp \left( - \frac{\ell^2}{2 \sigma_{\mathrm{s}}^2} \right) .
\end{equation*}
Its variance can be estimated using the sampled data.
An unbiased estimator for the variance is given by~\cite{zwillinger1995crc}
\begin{equation*}
\sigma_{\mathrm{s}}^2 = \frac{1}{N-1} \sum_{1}^{N} (l_{k}-\mu_{s})^2 ,
\end{equation*} 
where $N$ is the sample size, $l_{k}$ forms the data set, and points are expressed in the logarithmic domain \cite{kay1993fundamentals,poor1998introduction}.

In a practical setting, we need to estimate the wireless channel parameters $A$ and $B$.
To do so, we use the method of least squares, which minimizes the sum of the squares of the offsets.
Consider the linear least squares problem given by
\begin{equation*}
f(x, \beta) = \sum_{j = 1}^m \beta_j \phi_j(x) ,
\end{equation*}
where $\phi _{j}$ is a function of $x$ and $\beta$ is the parameter vector to be estimated.
Letting matrix $M$ be defined as
\begin{equation*}
M_{ij}= \frac{\partial f(x_i,\boldsymbol \beta)}{\partial \beta_j}
= \phi_j(x_{i}) ,
\end{equation*}
the least squares estimate for $\beta$ becomes
\begin{equation*}
\boldsymbol{\hat\beta} =( M^T M)^{-1} M^{T}\mathbf{y} .
\end{equation*}
For the least square problem defined in Section~\ref{channel}, we let 
\begin{xalignat*}{2}
\mathbf{y} &=
\begin{bmatrix*} p_{1} \\ \vdots \\ p_{N} \end{bmatrix*} &
M &=
\begin{bmatrix*}
1 & \log_{10}(d_{1})  \\ \vdots & \vdots \\ 1 & \log_{10}(d_{N}) \\
\end{bmatrix*}
\end{xalignat*}
where $\mathbf{y}$ denotes the vector of received powers.
Then, the estimator of $A$ and $B$ is given as
\begin{equation*}
\begin{bmatrix*} A \\ B \end{bmatrix*}
=( M^T M)^{-1} M^{T} \mathbf{y} .
\end{equation*}
The variance of $L_{s}$, the shadow fading component, is computed by normalizing the residual error
\begin{equation*}
\sigma_{\mathrm{s}}^2
= \frac{1}{N-1} \mathbf{y}^{T} (I-M( M^T M)^{-1} M^{T})\mathbf{y} .
\end{equation*}


\section{Problem Formulation}

Consider a scenario where several wireless devices are randomly positioned nearby an area of interest.
To simplify the problem, we assume the area to be rectangular shape.
Four RF monitoring devices are located at the corners of this region.
Each monitoring device has information concerning its own location and orientation.
The radiation pattern of the antenna attached to each monitoring device is known as well.
In our system model, all of the monitoring devices are connected to the Internet and send the captured data to a process center for inference.
Several wireless clients carried by users are randomly located near this area, they can be inside or outside the area of interest.
The wireless clients transmit data packets periodically and, consequently, they can be easily detected by the monitoring devices.
Since each wireless client has a unique MAC address, the packets transmitted from different clients can be distinguished.
Throughout, we use $\mathcal{A}_{\mathrm{t}}$ to represent the target area and $\mathcal{A}_{\mathrm{o}}$ to represent its complement.
A notional diagram of the framework is shown in Fig.~\ref{figure:NotionalDiagram}.
\begin{figure}[t]
\centerline{\begin{tikzpicture}
[draw=black, line width=0.5pt, inner sep=0pt,
sensor/.style={regular polygon, regular polygon sides=4, draw=black, fill=black, minimum size=6pt},
antenna/.style={ellipse, draw=none, fill=lightgray!50, minimum height=25pt, minimum width=35pt},
agent/.style={circle, draw=black, fill=black, minimum size=3pt},
outside/.style={circle, draw=black, minimum size=3pt}]

\draw[rounded corners] (0.5,0) rectangle (7.5,5);
\draw[rounded corners, dashed] (2.5,1) rectangle (5.5,4);

\node[sensor] (A1) at (2.5,1) {};
\node[sensor] (A2) at (5.5,1) {};
\node[sensor] (A3) at (5.5,4) {};
\node[sensor] (A4) at (2.5,4) {};

\node at (4.375,1.25) {$\mathcal{A}_{\mathrm{t}}$};
\node at (6,0.25) {$\mathcal{A}_{\mathrm{o}}$};


%\node[outside] at ( 2.1689410647 , 1.13775402812 ) {};
%\node[outside] at ( 2.20864248209 , 1.53164040373 ) {};
%\node[outside] at ( 2.21849793351 , 1.17616786499 ) {};
%\node[outside] at ( 2.22713611905 , 4.2883388606 ) {};
%\node[outside] at ( 2.33285623177 , 1.1314522837 ) {};
%\node[outside] at ( 2.34267400359 , 3.652875108 ) {};
\node[outside] at ( 2.47345299569 , 4.82783008639 ) {};
%\node[outside] at ( 2.51135159638 , 0.553624422059 ) {};
%\node[outside] at ( 2.52272453366 , 0.435755542295 ) {};
%\node[outside] at ( 2.89428515774 , 4.18216209512 ) {};
%\node[outside] at ( 2.49752853982 , 4.03759592194 ) {};
%\node[outside] at ( 2.51023990635 , 4.20218045094 ) {};
\node[agent] at ( 2.60516018122 , 2.05414997932 ) {};
%\node[agent] at ( 2.70214622891 , 3.55330240561 ) {};
%\node[outside] at ( 2.73317107458 , 0.543101220381 ) {};
%\node[agent] at ( 2.84935121658 , 1.15467928253 ) {};
\node[agent] at ( 2.83890584713 , 2.87152377485 ) {};
%\node[outside] at ( 2.96528485354 , 4.3464965053 ) {};
\node[outside] at ( 3.33708841304 , 0.44137061247 ) {};
\node[outside] at ( 3.19255851127 , 0.391147573008 ) {};
\node[agent] at ( 3.57135966442 , 1.25957249239 ) {};
\node[agent] at ( 3.57108996599 , 1.93263688797 ) {};
\node[agent] at ( 3.04428244813 , 2.09557253043 ) {};
\node[agent] at ( 3.65349360073 , 2.27916582855 ) {};
\node[agent] at ( 3.38165845495 , 2.65804427865 ) {};
\node[agent] at ( 3.801171656 , 2.95053880944 ) {};
\node[agent] at ( 3.52683781827 , 3.48204858325 ) {};
%\node[agent] at ( 3.11191070242 , 3.66641157636 ) {};
\node[agent] at ( 3.92046074421 , 3.81638294342 ) {};
\node[agent] at ( 3.95870651333 , 3.85414469965 ) {};
\node[agent] at ( 3.41861784707 , 3.90605027886 ) {};
\node[outside] at ( 3.25492339938 , 4.40184766792 ) {};
\node[outside] at ( 3.54457445854 , 4.54639692743 ) {};
\node[outside] at ( 3.95627386672 , 4.61065410754 ) {};
\node[outside] at ( 4.54334151472 , 0.318871863425 ) {};
\node[outside] at ( 4.61866285411 , 0.43085208228 ) {};
\node[outside] at ( 4.22256379727 , 0.810026849865 ) {};
\node[agent] at ( 4.11156281034 , 1.34890775483 ) {};
\node[agent] at ( 5.11503522961 , 1.73533325687 ) {};
\node[agent] at ( 4.25271922861 , 2.14289026089 ) {};
\node[agent] at ( 5.08072826862 , 2.20293685787 ) {};
\node[agent] at ( 4.39386727986 , 2.76582994285 ) {};
\node[agent] at ( 5.43175102407 , 2.82102596373 ) {};
\node[agent] at ( 4.17689929022 , 3.04555848331 ) {};
\node[agent] at ( 4.04574591112 , 3.10006143605 ) {};
\node[agent] at ( 4.02083927143 , 3.21271817124 ) {};
\node[agent] at ( 4.23913674945 , 3.31813256911 ) {};
%\node[agent] at ( 5.18350480449 , 3.91027292346 ) {};
\node[outside] at ( 4.7592722738 , 4.12835849149 ) {};
\node[outside] at ( 4.35389401606 , 4.17629790833 ) {};
%\node[outside] at ( 4.92106435495 , 4.30569934976 ) {};
\node[outside] at ( 4.56377465969 , 4.45205458561 ) {};
%\node[outside] at ( 5.39171408677 , 4.50063106357 ) {};
%\node[outside] at ( 5.38722123552 , 4.57106173982 ) {};
\node[outside] at ( 4.35846169798 , 4.67575131502 ) {};
%\node[outside] at ( 5.68900697359 , 4.61956803348 ) {};
%\node[outside] at ( 5.8012759141 , 0.786575025938 ) {};
\node[outside] at ( 0.683581674451 , 4.52341247559 ) {};
\node[outside] at ( 0.83059079898 , 2.52472319096 ) {};
\node[outside] at ( 0.787683048486 , 1.96310009868 ) {};
\node[outside] at ( 0.837916045306 , 0.306404502107 ) {};
%\node[outside] at ( 1.92026717467 , 3.93015374681 ) {};
%\node[outside] at ( 1.95515716802 , 4.18411686654 ) {};
\node[outside] at ( 1.81528797602 , 3.35173736185 ) {};
\node[outside] at ( 1.69019678097 , 3.6284536787 ) {};
\node[outside] at ( 1.14389478457 , 4.28971774147 ) {};
\node[outside] at ( 1.90302570449 , 3.19494880098 ) {};
\node[outside] at ( 1.2453267792 , 2.51959732965 ) {};
\node[outside] at ( 1.72075588007 , 0.334427306331 ) {};
\node[outside] at ( 6.68284689101 , 3.07494901779 ) {};
%\node[outside] at ( 6.31670524144 , 1.04082313737 ) {};
\node[outside] at ( 6.88778245976 , 4.49548052642 ) {};
%\node[outside] at ( 6.1509547887 , 1.46915491498 ) {};
\node[outside] at ( 6.04508822092 , 1.91410445168 ) {};
\node[outside] at ( 6.4571595723 , 2.68645486371 ) {};
\node[outside] at ( 6.44353320611 , 0.802163089791 ) {};
\node[outside] at ( 6.29718673575 , 3.68544388153 ) {};
\node[outside] at ( 6.97338238493 , 4.79375673213 ) {};
\node[outside] at ( 6.59506753066 , 1.82391001218 ) {};
\node[outside] at ( 6.77952439855 , 4.57765995441 ) {};
\node[outside] at ( 7.03774161105 , 1.75405750998 ) {};
\node[outside] at ( 7.1447155042 , 4.44145190374 ) {};
\node[outside] at ( 7.09582695689 , 3.58914971225 ) {};
\node[outside] at ( 7.11986402123 , 0.298547467349 ) {};
\node[outside] at ( 7.25334242268 , 2.50289493253 ) {};
\node[outside] at ( 7.13825615028 , 1.41391989682 ) {};
\end{tikzpicture}
}
\caption{The periphery of the target area is delineated by the dashed line. 
The squares at the corners of the target area denote the locations of the monitoring devices, each equipped with a directional antenna.
Clients within the zone of interest are in black, whereas outside agents appear in white.
The objective is to estimate the occupancy within target area.}
\label{figure:NotionalDiagram}
\end{figure}

In this thesis, we assume the wireless clients are quasi-static and each client is equipped with an isotropic antenna.
Thus, the orientation of the wireless clients does not matter.
For convenience, we use a single vector to denote the locations of the wireless clients:
\begin{equation}
\underline{\mathbf{U}} = (\mathbf{U}_1, \ldots, \mathbf{U}_{n_{\mathrm{a}}}) ,
\end{equation}
where $n_{\mathrm{a}}$ is the number of the detected clients.
We also assume that the signal captured by a monitoring device comes from a line-of-sight path.
Therefore, signal strength subscribes to a free-space transmission model.
The received signal strength from client $j$ to sensing device $i$ can be expressed using the log-normal channel model
\begin{equation} 
P_{ij} \text{[dBm]}
= A + B \log_{10}(d_{ij}) + L_{ij} + G_i (\phi_{ij}) ,
\end{equation}
where $A$ and $B$ are the mean decay parameters, $d_{ij}$ is the Euclidean distance between the client~$j$ and sensing device~$i$.
This distance is equal to
\begin{equation*}
d_{ij} = d(\mathbf{s}_i, \mathbf{u}_j)
= \sqrt{ (u_{1j} - s_{1i})^2 + (u_{2j} - s_{2i})^2 } .
\end{equation*}
Variable $L_{ij}$ represent shadow fading and parameter $G_i (\cdot)$ is the antenna gain function of the sensing device.
Parameter $\phi_{ij}$ denotes the angle of the signal transmission direction, which can be expressed as
\begin{equation*}
\phi_{ij} = \angle (\mathbf{s}_i, \mathbf{u}_j)
= \operatorname{atan2} ( u_{2j} - s_{2i}, u_{1j} - s_{1i} ) .
\end{equation*}
The shadow fading components $\{ L_{ij} \}$ are assumed to be independent and identically distributed log-normal random variables.
In the logarithmic domain, the corresponding probability density function becomes
\begin{equation} 
f_{L_{ij}} (\ell)
= \frac{1}{\sqrt{2 \pi} \sigma_{\mathrm{s}}} 
\exp \left( - \frac{\ell^2}{2 \sigma_{\mathrm{s}}^2} \right) ,
\end{equation}
where $\sigma_{\mathrm{s}}$ is the standard deviation of shadowing.

The observed information from the four sensing devices form a power matrix $\underline{\mathbf{P}} = (\mathbf{P}_1, \ldots, \mathbf{P}_{n_{\mathrm{a}}})$.
The vector element $\mathbf{P}_j = (P_{1j}, P_{2j},P_{3j},P_{4j})$ contains the signal strength of the wireless client~$j$ detected by four sensing devices.
We assume that the number and locations of the wireless clients located inside the area of interest form a Poisson point process with intensity $\lambda_{\mathrm{t}}$.
Therefore, the probability that $r_{t}$ wireless clients lie inside the target area is equal to
\begin{equation*}
\Pr ( R_{\mathrm{t}} = r_{\mathrm{t}} )
= \frac{(\lambda_{\mathrm{t}} A_{\mathrm{t}})^{r_{\mathrm{t}}}}
{r_{\mathrm{t}}!} e^{- A_{\mathrm{t}} \lambda_{\mathrm{t}}}
\quad r_{\mathrm{t}} = 0, 1, \ldots
\end{equation*}
where $R_{\mathrm{t}}$ is the number of clients inside and $A_{\mathrm{t}}$ is the area of the target region.
Similarly, we can write the probability that $r_{o}$ wireless clients are located outside the target area as
\begin{equation*}
\Pr ( R_{\mathrm{o}} = r_{\mathrm{o}} )
= \frac{(\lambda_{\mathrm{o}} A_{\mathrm{o}})^{r_{\mathrm{o}}}}
{r_{\mathrm{o}}!} e^{- A_{\mathrm{o}} \lambda_{\mathrm{o}}}
\quad r_{\mathrm{o}} = 0, 1, \ldots
\end{equation*}
where $R_{\mathrm{o}}$ is the number of clients outside, $A_{\mathrm{o}}$ is the area of the complement of target region, and $\lambda_{\mathrm{o}}$ is a Poisson intensity parameter.
The inference task is to estimate occupancy based on the power matrix data set $\underline{\mathbf{P}}$, which is collected by the sensing devices.


