\chapter{CONCLUSION}

In this thesis, we report two algorithms corresponding to different application scenarios for occupancy estimation using Wi-Fi monitoring.
We utilize NUCs with sensing antennas as monitoring devices.
We assess the performance of the estimators for isotropic and directional antennas through numerical simulations and prototyping.
Our results indicate that it is possible to accurately estimate the number of active agents within a prescribed area by deploying sensing devices about the area of interest.
Furthermore, performance is generally enhanced by the careful shaping of antenna radiation patterns.
That is, the performance of a monitoring system can be enhanced by employing a configuration that strongly discriminates between wireless agents that are located within and outside the target area.
In general, a more discriminating configuration yields considerable improvements over a generic setup with isotropic antennas.

This work can be extended for future research.
Potential problems include tracking occupancy over time, and estimate the density of people using non-uniform distributions.
It would also be interesting to use reconfigurable antennas to acquire very discriminating information about active devices for wireless inference.
Finally, a pragmatic goal would be to implement such a monitoring system indoors, where multipath fading exists and tracking the location of a particular device can be very challenging.
This information could then be used to dynamically adapt Wi-Fi access points to changing network traffic conditions.

