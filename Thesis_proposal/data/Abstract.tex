\TAMUAbstractFormat
%%%%%%%%%%%%%%%%%%%%%%%%%%%%%%%%%
%Please keep the line above and only type in below.
%%%%%%%%%%%%%%%%%%%%%%%%%%%%%%%%%

In various situations, there are demands for estimating the number of people in an specific area. This article focuses on estimating the number of device in a certain area based on Wi-Fi metadata. To accomplish this, four sensing devices are placed at the four points of a rectangular area respectively. The sensing devices can observe and record all local data packets under monitoring mode. For each sensing device, both directional and isotropic antennas are used to detect packets separately. Each sensing device retrieves the received signal strength indicators and the media access control  addresses from the 802.11 frames packets transmitted by the active wireless devices nearby.
The estimator takes received signal strength indicators as input and infers the number of active Wi-Fi devices inside the specific area. Two algorithms, bayesian and maximum-likelihood are employed for the schemes.
We also compare the performances between directional antenna and isotropic antenna. The result shows by using the directional monitoring antennas, we can obtain better accuracy.
