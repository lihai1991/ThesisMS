\chapter[INTRODUCTION]{INTRODUCTION}
The number of Wi-Fi access points and the number of Wi-Fi client devices have been increasing in recent years. According to Cisco Systems prediction in their Visual Network Index, 55 percent of total mobile data traffic will be offloaded onto fixed networks through Wi-Fi access points and femtocells by 2020 \cite{CiscoVNI2016}. Modern smartphones with Wi-Fi module transmit Wi-Fi messages periodically. Therefore, it provides us a chance to estimate occupancy through Wi-Fi packets. By deploying Wi-Fi monitoring devices in an area of interest, it is possible to detect these Wi-Fi transmissions. Each Wi-Fi transmission contains MAC address and RSSI. The MAC address is the identifier of the device while RSSI indicates the physical distance between transmitter and receiver. In the research, we focus on occupancy estimation based on Wi-Fi packets and analyze the benefits using specific directional antenna.
In this paper, we apply Bayes estimation scheme and maximum likelihood scheme to estimate occupancy. A numerical simulation based on these two schemes shows the performances corresponding to sensing devices with directional antennas and isotropic antennas. In addition, to evaluate the simulation results, we construct an experimental environment in 100j parking lot. In the experiment we use four sensing devices and deploy them in the four corners of area of interest, respectively.
The rest of this paper is organized as below. In section 2, we explain related work. In section 3, we introduce the problem formulation. In section 4, we propose two estimation schemes. In section 5, we give the simulation results of performance. In section 6, we show the experiment result along with description of experiment setup. Finally we conclude this paper in section 7.

 






 


